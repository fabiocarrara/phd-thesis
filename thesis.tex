% For printing in a4
\documentclass[a4,10pt,twoside,openright,italian,english]{book}% twoside!

% For printing with the A5 format
%\documentclass[10pt,twoside,openright,english,italian]{book}% twoside!

% Set paper size
\usepackage[twoside=true]{geometry}

%For printing with the weird format
%\geometry{
%	paperwidth=17cm,
%	paperheight=24cm,
%	margin=2cm,
%	top=2.3cm,
%	bindingoffset=0.4cm
%}
% For printing in a4
\geometry{a4paper,
  margin=3cm,
  top=3.8cm,
  bindingoffset=0.4cm
}

%Uncomment this for final prints: this just enables printing on a4 paper
\usepackage[cam,center,a4,pdflatex,axes]{crop}

\usepackage{phdthesis}

%\usepackage{fancyhdr}
\usepackage{color}
\usepackage{array}
\usepackage{mdwmath}
\usepackage{mdwtab}
\usepackage{amsmath,amssymb}
\usepackage{cite}
%\usepackage{graphicx}
\usepackage{listings}
\usepackage{subfig}
\usepackage{booktabs}
\usepackage{latexsym}
%\usepackage{color}
\usepackage{url}
\usepackage{bnf}
\usepackage{rotating}
\usepackage{multirow}
\usepackage{phdtitle}
\usepackage{paralist}
\usepackage{bibentry}
%\usepackage[algochapter]{algorithm2e}
\usepackage[bookmarks=true,
%pdftex=false,
bookmarksopen=true,hidelinks]{hyperref}
%\usepackage[toc,acronym]{glossaries}
\usepackage{lscape}
\usepackage{algorithmic}
\usepackage{algorithm}
\usepackage{longtable}
\usepackage[T1]{fontenc}
%\usepackage[latin1]{inputenc}
\usepackage[utf8]{inputenc}
%\usepackage{fontspec}
%\setmainfont{Calibri}
%% \hyphenation{} is used to force the
\hyphenation{}

\newtheorem{Definition}{Definition}[section]
\DeclareMathOperator*{\argmin}{arg\,min}

\lstset{tabsize=2,basicstyle=\footnotesize,breaklines=true}

\usepackage[main=english,italian]{babel}
\usepackage{ulem}
\normalem

\hypersetup{pdftitle={Title}, pdfauthor={Author}}

\nobibliography*

\department {Dottorato di ricerca in Ingegneria dell'Informazione}

% Please fulfil the followed fields with your data
\author{Fabio Carrara}
\title{Deep Learning for Image Classification and Retrieval: Opportunities and Limitations}
\tutor{Tutor Name}
\supervisor{Coordinator Name}
\titleimage{img/unipi.png} %please not change
\phdyear{Year}
\phdmonth{Month}
\phdcycle{Cycle XXXI}

%%%%%%%%%%%%%%%%%%%%%%%%%%%%%%%%%%%%%%
% Let's Start The Real Document
%%%%%%%%%%%%%%%%%%%%%%%%%%%%%%%%%%%%%%
\newglossaryentry{matrix_channel}
{       name={$H^*$},
        description={Conjugate operation}
}
\newglossaryentry{trasp_x}
{       name={$[ x ]^{\rm T}$},
        description={transpose operator}
}
\newglossaryentry{vec_x}
{       name={\textbf{x}},
        description={vectors are in bold}
}
\newglossaryentry{floor_funct}
{       name={$\left\lfloor x \right\rfloor$},
        description={round to the lower integer of $x$}
}

% Deep Learning
\newacronym{ml}{ML}{Machine Learning}
\newacronym{dl}{DL}{Deep Learning}
\newacronym{dnn}{DNN}{Deep Neural Network}
\newacronym{ffnn}{FFNN}{Feed-Forward Neural Network}
\newacronym{rnn}{RNN}{Recurrent Neural Network}
\newacronym{mlp}{MLP}{Multilayer Perceptron}
\newacronym{cnn}{CNN}{Convolutional Neural Network}
\newacronym{lstm}{LSTM}{Long Short-term Memory}
\newacronym{sgd}{SGD}{Stochastic Gradient Descent}
\newacronym{adam}{Adam}{Adaptive Moment Estimation}
\newacronym{relu}{ReLU}{Rectified Linear Unit}

\newacronym{bn}{BN}{Batch Normalization}
\newacronym{rmac}{R-MAC}{Regional Maximum Activation of Convolutions}

% CBIR
\newacronym{cbir}{CBIR}{Content-based Image Retrieval}
\newacronym{map}{mAP}{mean Average Precision}
\newacronym{pca}{PCA}{Principal Component Analysis}


% Datasets
\newacronym{ilsvrc}{ILSVRC}{ImageNet Large Scale Visual Recognition Challenge}

% Others
\newacronym{gpu}{GPU}{Graphical Processing Unit}
\newacronym{lbp}{LBP}{Local Binary Patterns}
\newacronym{lpq}{LPQ}{Local Phase Quantization} % CHECK
\newacronym{str}{STR}{Surrogate Text Representation}
\newacronym{svm}{SVM}{Support Vector Machine}



\makeglossaries

\begin{document}
\selectlanguage{english}

\maketitle

\pagestyle{empty}

\cleardoublepage
\newpage

%%%%%%%%%%%%%%%%%%%%%%%%%%%%%%%%%%%%%%
% Dedication - For removing dedication, please comment or delete the code inside \thispagestyle{empty}
%%%%%%%%%%%%%%%%%%%%%%%%%%%%%%%%%%%%%%
\thispagestyle{empty}
    \null\vspace{\stretch {1}}
        \begin{flushright}
                This thesis is dedicated to....
        \end{flushright}
\vspace{\stretch{2}}\null

\cleardoublepage
\newpage

\pagestyle{empty}
%% change numbering into Roman numbers for the introductory part
%\setcounter{page}{1}
%\pagenumbering{Roman}

%%%%%%%%%%%%%%%%%%%%%%%%%%%%%%%%%%%%%%
% Quotes - For removing quotes, please comment or delete the code inside \thispagestyle{empty}
%%%%%%%%%%%%%%%%%%%%%%%%%%%%%%%%%%%%%%
\thispagestyle{empty}
    \null\vspace{\stretch {1}}
        \begin{flushright}
                "A few first rate research papers are preferable to a large number \\
                that are poorly conceived or half-finished.\\
                The latter are no credit to their writers and \\
                a waste of time to their readers"\\
                Claude Shannon
                %IRE Transactions on Information Theory (1956), volume 2, issue 1, page 3. Shannon, Claude E. (March 1956), The Bandwagon, 2, doi:10.1109/TIT.1956.1056774.
        \end{flushright}
\vspace{\stretch{2}}\null

\cleardoublepage
\newpage

\pagestyle{empty}
%% change numbering into Roman numbers for the introductory part
\setcounter{page}{1}
\pagenumbering{Roman}

%%%%%%%%%%%%%%%%%%%%%%%%%%%%%%%%%%%%%%
% Acknowledgement
%%%%%%%%%%%%%%%%%%%%%%%%%%%%%%%%%%%%%%
\chapter*{Acknowledgements}
\lettrine{A}{cknowledgements} goes here.

\selectlanguage{italian}
\chapter*{Ringraziamenti}
\lettrine{R}{ingraziamenti} 
\selectlanguage{english}

\cleardoublepage
\newpage

\pagestyle{fancy}
% change numbering into Roman numbers for the introductory part

%%%%%%%%%%%%%%%%%%%%%%%%%%%%%%%%%%%%%%
% Summary
%%%%%%%%%%%%%%%%%%%%%%%%%%%%%%%%%%%%%%
\selectlanguage{english}
\chapter*{Summary}
\lettrine{T}{he} large diffusion of cheap cameras and smartphones led to an exponential daily production of digital visual data, such as images and videos.
In this context, most of the produced data lack manually assigned metadata needed for their manageability in large-scale scenarios, thus shifting the attention to the automatic understanding of the visual content.
Recent developments in Computer Vision and Artificial Intelligence empowered machines with high-level vision perception enabling the automatic extraction of high-quality information from raw visual data.
Specifically, \acrfullpl{cnn} provided a way to automatically learn effective representations of images and other visual data showing impressive results in vision-based tasks, such as image recognition and retrieval.

In this thesis, we investigated and enhanced the usability of \acrshortpl{cnn} for visual data management.
First, we identify three main limitations encountered in the adoption of \acrshortpl{cnn} and propose general solutions that we experimentally evaluated in the context of image classification.
We proposed miniaturized architectures to decrease the usually high computational cost of \acrshortpl{cnn} and enable edge inference in low-powered embedded devices.
%We evaluated our proposal in a practical distributed application, i.e., visual parking lot occupancy detection, extensively comparing the reduced models to state-of-the-art methods and architectures.
We tackled the problem of manually building huge training sets for models by proposing an automatic pipeline for training classifiers based on cross-media learning and Web-scraped weakly-labeled data.
We analyzed the robustness of \acrshortpl{cnn} representations to out-of-distribution data, specifically the vulnerability to adversarial examples, and proposed a detection method to discard spurious classifications provided by the model.
%
Secondly, we focused on the integration of \acrshort{cnn}-based \acrfull{cbir} in the most commonly adopted search paradigm, that is, textual search.
We investigated solutions to bridge the gap between image search and highly-developed textual search technologies by reusing both the front-end (text-based queries) and the back-end (distributed and scalable inverted indexes). % in \acrshort{cbir} scenarios.
We proposed a cross-modal image retrieval approach which enables textual-based image search on unlabeled collections by learning a mapping from textual to high-level visual representations.
Finally, we formalized, improved, and proposed novel surrogate text representations, i.e., text transcriptions of visual representations that can be indexed and retrieved by available textual search engines enabling \acrshort{cbir} without specialized indexes.

\selectlanguage{english}

%%%%%%%%%%%%%%%%%%%%%%%%%%%%%%%%%%%%%%
% Italian Summary
%%%%%%%%%%%%%%%%%%%%%%%%%%%%%%%%%%%%%%

%There must be an Italian version of the summary
\selectlanguage{italian}
\chapter*{Sommario}
\lettrine{S}{ommario} va qui.


\selectlanguage{english}

%%%%%%%%%%%%%%%%%%%%%%%%%%%%%%%%%%%%%%
% Publications
%%%%%%%%%%%%%%%%%%%%%%%%%%%%%%%%%%%%%%

%List of publications of the PhD candidate
\selectlanguage{english}
\chapter*{List of publications}

% for citing your paper, take directly APA format from Google scholar
% \item Surname, N., Surname, N. and Surname, N. (Year,Month). Title of the paper or journal. \emph{Place of publication}. (Vol. 123, pp. 456). EditorName.

\section*{International Journals}
\begin{enumerate}
    \item Amato, G., Carrara, F., Falchi, F., Gennaro, C., Meghini, C., and Vairo, C. (2017, April). Deep learning for decentralized parking lot occupancy detection. \emph{Expert Systems with Applications}. (Vol. 72, pp. 327-334). Pergamon.
    \item Carrara, F., Esuli, A., Fagni, T., Falchi, F., and Fernández, A. M. (2018, June). Picture it in your mind: Generating high level visual representations from textual descriptions. \emph{Information Retrieval Journal}. (Vol. 21, pp. 208-229). Springer.
    % manca mese, volume e pagine
    \item Carrara, F., Falchi, F., Caldelli, R., Amato, G., and Becarelli, R. (2018). Adversarial image detection in deep neural networks. \emph{Multimedia Tools and Applications}. (pp. 1-21). Springer.
\end{enumerate}

\section*{International Conferences/Workshops with Peer Review}
\begin{enumerate}
    \item Amato, G., Carrara, F., Falchi, F., Gennaro, C., and Vairo, C. (2016, June). Car parking occupancy detection using smart camera networks and deep learning. In \emph{2016 IEEE Symposium on Computers and Communication (ISCC)}. (pp. 1212-1217). IEEE.
    \item Carrara, F., Falchi, F., Caldelli, R., Amato, G., Fumarola, R., and Becarelli, R. (2017, June). Detecting adversarial example attacks to deep neural networks. In \emph{Proceedings of the 15th International Workshop on Content-Based Multimedia Indexing (CBMI)}. (p. 38). ACM.
    \item Amato, G., Carrara, F., Falchi, F., and Gennaro, C. (2017, June). Efficient Indexing of Regional Maximum Activations of Convolutions using Full-Text Search Engines. In \emph{Proceedings of the 2017 ACM on International Conference on Multimedia Retrieval (ICMR)}. (pp. 420-423). ACM.
    \item Vadicamo, L., Carrara, F., Cimino, A., Cresci, S., Dell’Orletta, F., Falchi, F., and Tesconi, M. (2017, October). Cross-Media Learning for Image Sentiment Analysis in the Wild. In \emph{2017 IEEE International Conference on Computer Vision (ICCV) Workshops}. (pp. 308-317).
    \item Amato, G., Bolettieri, P., Carrara, F., Falchi, F., Gennaro, C. (2018, June). Large-Scale Image Retrieval with Elasticsearch. In \emph{Proceedings of the 41st International ACM Conference on Research and Development in Information Retrieval (SIGIR)}. (pp.  925-928). ACM.
    \item Messina, N., Amato, G., Carrara, F., Falchi, F., and Gennaro, C. (2018, September). Learning Relationship-aware Visual Features. To appear in \emph{2018 IEEE European Conference on Computer Vision (ECCV) Workshops}.
    \item Carrara, F., Becarelli, R., Caldelli, R., Falchi, F. and Amato, G. (2018, September). Adversarial examples detection in features distance spaces. To appear in \emph{2018 IEEE European Conference on Computer Vision (ECCV) Workshops}.
\end{enumerate}

%\section*{Others}
%\begin{enumerate}
%    % for citing your paper, take directly APA format from Google scholar
%    \item Surname, N., Surname, N. and Surname, N. (Year,Month). Title of the paper or journal. \emph{Place of publication}. (Vol. 123, pp. 456). EditorName.
%\end{enumerate}

\selectlanguage{english}

%%%%%%%%%%%%%%%%%%%%%%%%%%%%%%%%%%%%%%
% List of Abbreviation and Symbols
%%%%%%%%%%%%%%%%%%%%%%%%%%%%%%%%%%%%%%
%Have a look at \gls command... you can refer a term in multiple ways
\printglossary[type=\acronymtype,title=List of Abbreviations]
\let\cleardoublepage\clearpage
\printglossary[title=Notation]
\cleardoublepage
\newpage
%%%%%%%%%%%%%%%%%%%%%%%%%%%%%%%%%%%%%%
% TOC
%%%%%%%%%%%%%%%%%%%%%%%%%%%%%%%%%%%%%%
\tableofcontents
\cleardoublepage
\newpage

% Now lets go back to normal numbering
\setcounter{page}{1}
\pagenumbering{arabic}

\cleardoublepage
%============================= INTRODUCTION =================================

\chapter{Introduction}
\label{ch:introduction}

% . vision probably the most important sense, convey much info fast
Vision is one of the primary senses of human beings, if not the most important one.
From generic signage to advertisements and artistic photography, imagery is ubiquitous in our lives due to its ability to convey lots of information quickly while overcoming cultural and linguistic barriers.
% . photos and videos everyday part of our lives, lots of data % -- ~13.37 img/day rita
Thanks to the ease of access to camera-equipped smartphones and social networks, we collectively generate, share, and receive a ridiculous amount of digital photos and videos daily.
According to InfoTrends\footnote{\url{http://blog.infotrends.com/}}, the number of digital photos taken worldwide in 2017 estimated at around 1.3 trillion, and the pace of digital media creation is intended to grow as more people get access to the Internet and cheap camera technology.
% -- problema organizzazione --> understanding automatico
% -- parallelo SE per testo <-> immagini
In lights of this scenario, there is an increasing interest in creating automatic tools for the management of digital visual data pursuing the same accessibility revolution textual search engines brought to the World Wide Web in the 90s and 00s.
Despite manual annotation and captioning of images and videos helped to build successful systems (e.g., keyword-based image search engines), methods relying on metadata surrounding visual data --- such as keywords, tags, captions --- require the creation of such metadata by human actors, which is a labor-intensive and subjective task that cannot scale to the current trend of digital media creation.
% on scales ranging from personal photo collections to big data companies with millions of users.
To overcome these limitations, the attention shifted to methods that try to model and infer the visual semantics in imagery relying solely upon the visual content, i.e., the information that machines can automatically extract from raw pixels and store it in numerical representations (image descriptors).
Early attempts to create image descriptors rely on low-level manually defined features of the images, such as the distribution of edges, colors, simple shapes, to name a few.
Most of the effort in these solutions is focused on manually defining the right combination of low-level features performing well for the specific task under analysis, which requires a considerable amount of domain expertise.

% . AI for vision in last years exploded
% -- img repre
% -- hand-crafted prima del 2012
% -- convnet e end2end grazie a ...
In the last years, a new wave in a field of Artificial Intelligence, called Deep Learning, enabled researchers to automatize perception and understanding of visual data by extracting information with high-level of abstraction from raw pixels, drastically limiting the human intervention in the process.
With the term Deep Learning, the research community indicates the family of Machine Learning techniques which aim to automatically learn from data a hierarchy of features extractors which map the input data in a high-level feature space tailored to a specific task to solve.
In the context of computer vision and image representation, Deep Learning, and in particular Convolutional Neural Networks, revolutionized feature engineering and visual understanding, outperforming handcrafted models on multiple vision tasks such as object recognition and detection, image description and retrieval,  and many more.
Convolutional Neural Networks are artificial neural networks specifically tailored to process image data and trained in a supervised end-to-end fashion.
Although CNNs have been around for many years, we pinpoint their turning point in 2012, year in which a deep convolutional neural network model outperformed approaches based on handcrafted features in the \acrlong{ilsvrc}.
Following this trend, the last six years experienced an overwhelming adoption of deep neural network models which set the new state of the art in numerous applications spanning multiple fields, including visual perception and image understanding.
This reborn of CNNs is attributed to multiple factors, the most important being the availability of large-scale datasets of labeled images (such as ImageNet) and the computational power offered by modern hardware accelerators such as GPUs.
Both of these factors contributed to training bigger models with millions of parameters which can achieve astonishing performance in challenging problems, such as fine-grained image classification including thousands of high-level concepts and semantics.

Nevertheless, deep-learning-based solutions pose non-trivial engineering challenges in their adoption.
In order to learn a functional hierarchy of features, models are defined to be deep, i.e., need to stack many parametric transformations (also called layers).
This not only considerably increases the computational budget for the model evaluation, but also increases the amount of supervision (in terms of the size of labeled data) needed to learn the parameters of the model properly.
The high computational budget drastically limits the applications of deep learning solutions in restricted environments with limited power resources, such as IoT devices and smartphones, which currently delegate complex data analysis to a centralized server.
Concerning training data, even if its creation is a one-time process, the manual labeling needed for its preparation still represents one of the highest cost of this kind of solutions.

In this thesis, ...
% . this thesis
% -- simplest formulation of image understanding (classif) -- limits
% --- engineering solutions to practical CNN problems in classif
% -- briding gap between technology of text and image search




%============================= BACKGROUND =================================

\chapter{Background}
\label{ch:background}

In this chapter, we present the basic concepts about \gls{dl} and an overview of the research fields on which its application has been investigated in this thesis, namely Image Classification and \gls{cbir}.
The chapter is organized as follows.
In Section~\ref{sec:back:deep-learning}, we provide the reader with a quick introduction to \gls{dl}, focusing on deep neural networks for image and text processing and gradient-based optimization.
In Section~\ref{sec:back:image-classification}, an introduction to image classification using convolutional neural networks is presented together with a review of successful approaches in this field.
In Section~\ref{sec:back:image-retrieval}, we describe the main aspects of \gls{cbir} based on image representations extracted from deep neural networks, and we discuss some state-of-the-art methodologies to build effecive description of images and to efficiently index them in large-scale scenarios.
Section~\ref{sec:back:datasets} summarizes the public datasets used in the experiments presented in this thesis.


\section{Deep Learning}
\label{sec:back:deep-learning}

\acrfull{dl} defines the set of \gls{ml} methods aiming to learn from data a \emph{hierarchy of representations} specialized for the task under consideration~\cite{goodfellow2016deep}.
\gls{dl} models are usually organized as a sequence (or more generally a graph) of parametric non-linear transformations, known as \emph{layers}, that acts like features extractors;
starting from raw data, each layer searches for useful patterns in its input and provides higher-level representation of the data to the next layer.
More formally, given an input $\mathbf{x}$ and $L$ non-linear transformations $f_l(\cdot; \theta_l)$ parametrized by $\theta_l$ ($l=1, \dots, L$), we can express the output \mathbf{y} of the cascade of transformations as:

\begin{align} \label{eq:back:deepnet}
    \mathbf{y} & = f(\mathbf{x}, \Theta) \\
               & = f_L(\dots f_2(f_1(\mathbf{x}; \theta_1); \theta_2); \theta_L)
\end{align}

where $\Theta = \{\theta_l, l = 1, \dots L\}$ indicates the set of all parameters, also known as \emph{weights}.

Given a training set $\mathbf{X} = \{(\mathbf{x}_i, \mathbf{y^\star}_i), i=1,\dots,N\}$ comprised by $N$ couples of inputs and desired outputs, the quality of a particular setting of parameters is quantitatively defined by a \emph{loss function} $\mathcal{L}(X; \Theta)$ that measures how much predictions and targets differ;
the loss function is usually defined as the average of the individual loss values computed on each sample of the dataset:

\begin{align}
    \mathcal{L}(X; \Theta) &= \frac{1}{N} \sum_{i=1}^N \mathcal{L}(\mathbf{y}_i, \mathbf{y^\star}_i) \\
                           &= \frac{1}{N} \sum_{i=1}^N \mathcal{L}(f(\mathbf{x}_i; \Theta), \mathbf{y^\star}_i)
\end{align}

where the particular formulation of $\mathcal{L}(\mathbf{y}_i, \mathbf{y^\star}_i)$ is task-dependent and further discussed in Section~\ref{}. % TODO
%In the learning phase, the model is optimized by changing the parameters $\Theta$ in such a way that the output $\mathbf{y}$ reflects the desired target $\mathbf{y^\star}$.
The learning problem is an optimization problem in which we search the best parameter setting $\Theta^\star$ that minimizes the loss function $\mathcal{L}(X; \Theta)$:

\begin{equation} \label{eq:back:optim}
    \Theta^\star = \argmin_\Theta \mathcal{L}(X; \Theta)
\end{equation}

For historical reasons, \gls{dl} models are also referred to as \emph{\glspl{dnn}} due to the resemblance of layers and their organization to the way neurons are interconnected and organized in the mammalian brain~\cite{}.  % TODO
\Glspl{dnn} can be roughly categorized in \glspl{ffnn}, in which information flows from input to output in a non-recursive cascade of computations, and \glspl{rnn}, which present a feedback loop in their computation graph.
In the following sections, we will review some practical and successful formulations of \glspl{dnn} in terms of their layers
%that are useful when dealing with image data
, and we will provide the reader with the basics of gradient-based optimization of~\eqref{eq:back:optim}.

\subsection{Feed-Forward Neural Network}
\label{subsec:back:ffnn}

\Acrlongpl{ffnn} are \gls{dl} models whose computation graph can be expressed as a directed acyclic graph, i.e.\,there are no feedback loops and information flows from inputs to outputs in a cascade fashion.
Thus, when computing of the whole chain from inputs to outputs, called the \emph{forward} pass of the network, each transformation defined by layers is computed only once.

In the following, we summarize some of the most relevant layers used in \glspl{dnn}.

\subsubsection{Fully Connected Layer}

The Fully Connected (or Inner Product) layer is a basic building block for \glspl{dnn}.
It performs a linear projection of the input followed by a usually non-linear element-wise activation function.
Formally, the output  $\mathbf{y} \in \mathcal{R}^m$ of the layer is obtained as follows:

\begin{equation}
    \mathbf{y} = \varphi ( \mathbf{W} \mathbf{x} + \mathbf{b} )
\end{equation}

where $\mathbf{x} \in \mathcal{R}^n$ is the input data, $\mathbf{W} \in \mathcal{R}^{n \times m}$ and $\mathbf{b} \in \mathcal{R}^m$ are the parameters of a linear projection to a $m$-dimensional space.
Usual choices for the activation function $\varphi: \mathcal{R} \to \mathcal{R}$ are the \gls{relu}, the sigmoid $\sigma$, or $\tanh$ functions (see Figure~\ref{}). % TODO figures of activations

% TODO FC as M perceptrons and biological discussion
% The weights in an artificial neural networks are interpreted as the strength of the interconnection between neuron cells

%MLP
%Convolutional Neural Networks
%Recurrent Neural Networks
%LSTMs (Bidir)
%Loss Functions
%Cross-entropy
%MSE
%L2 Weight decay
%Gradient-Based Optimization
%Backprop
%Optimizers: SGD (with momentum) / Adam
%Dropout (here?)

\section{Image Classification}
\label{sec:back:image-classification}

%Problem Setting
%Single-label binary- and multi-class image classification problem
%Examples (simple classif., sentiment analysis, etc.)
%Evaluation Metrics
%Top-k Accuracy
%AUC of ROC (TPR, FPR, confusion matrix)
%Recent Advances
%ILSVRC winners (Hybrid, VGG, Inception, Residual, ResNeXT, SENets?)
%Transfer Learning

%% probably move in the chapter
%Adversarial Examples for DNNs
%Definition
%Adv example formal definition
%Properties
%Adv. Generation Algorithms
%L-BFGS
%FGSM
%etc.
%Defense Strategies
%Change the net: Adversarial Training / other..
%Detect attacks: some rel. works on that


\section{Image Retrieval}
\label{sec:back:image-retrieval}

%Problem Setting
%CBIR (query-by-example)
%kNN schemes
%Image Representations
%Deep Features (fc7 -> RMAC)
%Permutation-based representations
%Deep Permutations
%Cross-media Retrieval
%Textual / visual / common space retrievals
%Datasets & Evaluation Metrics
%mAP, R@K, nDCG, medR, MRR

\setion{Datasets}
\label{sec:back:datasets}
%ILSVRC (+Places)?, (PKLot, CNRPark)? TwitTestDataset? T4SA?
%Holidays, Oxford, Paris + distr, COCO


\cleardoublepage
\phantomsection
\addcontentsline{toc}{chapter}{\bibname}
\small
\bibliographystyle{plain}
\bibliography{thesis}

\end{document}
